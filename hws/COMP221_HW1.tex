\documentclass{exam}
\usepackage{graphicx} % Required for inserting images
\usepackage{algorithmicx}
\usepackage{algpseudocode}
\usepackage{geometry}[border=1in]
\usepackage{algorithm}
\usepackage{amsmath}
\usepackage{amssymb}
\usepackage{listings}
\usepackage{mathtools}
\usepackage{hyperref}
\DeclarePairedDelimiter\ceil{\lceil}{\rceil}
\DeclarePairedDelimiter\floor{\lfloor}{\rfloor}

\printanswers

\title{Homework 1: Big-O}
\author{COMP221 Spring 2025 - Suhas Arehalli}
\date{}

\begin{document}

\maketitle

Complete the problems below. Note that point values are roughly inversely correlated with expected difficulty.

\begin{itemize}
    \item Write up your solutions to all parts of this homework so that they may be exported to a typeset PDF (using LaTeX with a tool like Overleaf is strongly recommended).
    \item You'll submit this assignment through Moodle.
    \item The due date for this assignment will be posted on the course website.
    \item If you discuss problems with other students, list their names at the top of your assignment (there will be no penalty for this --- it's encouraged!).
    \item However, don't look at other group's written solutions, or share your written solutions with other groups. Consult the syllabus for more details on academic integrity policies! You should make sure your groups are sharing \textit{approaches} to a problem, not full solutions. 
\end{itemize}   

\section*{Problems} 
\begin{questions}
    \question \textbf{Ignoring coefficients} (\textit{15pts}):
    In Data Structures, we learned that we can drop leading coefficients in time complexities when using Big-O notation. A version of this claim can be written formally as the following: 
    
    If $f(n) \in \Theta(g(n))$, then for any $k > 0$, $kf(n) \in \Theta(g(n))$ 
    
    Prove this to be correct using the $c, n_0$ definitions of big-$O/\Omega/\Theta$.

    \textbf{HINT}: \textit{Try and generalize from more concrete examples. Is $3n \in O(n)$? Is $3n^2 \in O(n^2)$? What about $9n^2$? Can you see a way to construct the $c$ and $n_0$ to prove $9n^2 \in O(n^2)$ from the $c'$ and $n_0'$ that show $3n^2 \in O(n^2)$?} 

    \question \textbf{Counting for Quadratics} (\textit{10pts})
    In class, I was often actively sloppy when counting simple operations of $\Theta(n^2)$ sorts, suggesting that being off by a constant amount of operations doesn't matter. In these kinds of nested-loop algorithms, I claim that we will get a time complexity of the form
    \begin{align*}
        f(n) = a_1n^2 + a_2n + a_3
    \end{align*}
    where $a_1, a_2, a_3 \in \mathbb{Z}$ (i.e., they're integers) and $a_1 > 0$ --- a quadratic function with a positive $n^2$ term --- regardless of how much I miscount. 
    
    \textbf{Prove that for \textit{any} integers $a_1 > 0, a_2, a_3$, $f(n) \in \Theta(n^2)$.} Use the $c, n_0$ definitions (i.e., no limits). If helpful, you may assume that $f(n) \geq 0$ for all $n > 0$.

    \textbf{HINT:} \textit{Try and get the quadratic to be something easily factorable. Try by cases: If $a_3 = 0$, if $a_3 > 0$, or if $a_3 < 0$.}


    \question \textbf{Polynomials of Higher Degree} (\textit{5pts})
    Of course, the above only shows that this works for quadratic algorithms. Let's extend it to all polynomials: Show that if
    \begin{align*}
        f(n) &= \sum_{i=0}^k a_in^i \\
             &= a_kn^k + \dots + a_1n + a_0
    \end{align*}
    for some $a_i \in \mathbb{Z}$, then $f(n) \in O(n^k)$. Use the $c, n_0$ definition of big-O. Note I'm only asking for big-$0$ here, not big-$\Theta$!

    \textbf{HINT}: We'll need a slightly different strategy (one more general than the small bounding trick I suggested for quadratics). Consider $c = \sum_{i=0}^k |a_i|$. Can you see a way to show $cn^k \geq f(n)$ for all $n \geq$ some $n_0$?
\end{questions}

\end{document}